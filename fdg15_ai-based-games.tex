% This is "sig-alternate.tex" V2.0 May 2012
% This file should be compiled with V2.5 of "sig-alternate.cls" May 2012
%
% This example file demonstrates the use of the 'sig-alternate.cls'
% V2.5 LaTeX2e document class file. It is for those submitting
% articles to ACM Conference Proceedings WHO DO NOT WISH TO
% STRICTLY ADHERE TO THE SIGS (PUBS-BOARD-ENDORSED) STYLE.
% The 'sig-alternate.cls' file will produce a similar-looking,
% albeit, 'tighter' paper resulting in, invariably, fewer pages.
%
% ----------------------------------------------------------------------------------------------------------------
% This .tex file (and associated .cls V2.5) produces:
%       1) The Permission Statement
%       2) The Conference (location) Info information
%       3) The Copyright Line with ACM data
%       4) NO page numbers
%
% as against the acm_proc_article-sp.cls file which
% DOES NOT produce 1) thru' 3) above.
%
% Using 'sig-alternate.cls' you have control, however, from within
% the source .tex file, over both the CopyrightYear
% (defaulted to 200X) and the ACM Copyright Data
% (defaulted to X-XXXXX-XX-X/XX/XX).
% e.g.
% \CopyrightYear{2007} will cause 2007 to appear in the copyright line.
% \crdata{0-12345-67-8/90/12} will cause 0-12345-67-8/90/12 to appear in the copyright line.
%
% ---------------------------------------------------------------------------------------------------------------
% This .tex source is an example which *does* use
% the .bib file (from which the .bbl file % is produced).
% REMEMBER HOWEVER: After having produced the .bbl file,
% and prior to final submission, you *NEED* to 'insert'
% your .bbl file into your source .tex file so as to provide
% ONE 'self-contained' source file.
%
% ================= IF YOU HAVE QUESTIONS =======================
% Questions regarding the SIGS styles, SIGS policies and
% procedures, Conferences etc. should be sent to
% Adrienne Griscti (griscti@acm.org)
%
% Technical questions _only_ to
% Gerald Murray (murray@hq.acm.org)
% ===============================================================
%
% For tracking purposes - this is V2.0 - May 2012

\documentclass{sig-alternate}

\usepackage{graphicx}
\usepackage{amsmath}
\usepackage[skip=5pt]{caption}
%\usepackage{hyperref}
\usepackage{enumitem}



\newcommand{\todo}[1]{\textbf{[[#1]]}}
\newcommand{\pseudosection}[1]{\vspace{0.5\baselineskip} \noindent {\bf #1}}


\begin{document}
% --- Author Metadata here ---
\conferenceinfo{Foundations of Digital Games}{2015 Asilomar Conference Grounds, California USA}
%\CopyrightYear{2007} % Allows default copyright year (20XX) to be over-ridden - IF NEED BE.
%\crdata{0-12345-67-8/90/01}  % Allows default copyright data (0-89791-88-6/97/05) to be over-ridden - IF NEED BE.
% --- End of Author Metadata ---

\title{AI-based Games: Contrabot and What Did You Do?}


\numberofauthors{1}
\author{
\alignauthor
anonymous
%Michael Cook, Brent Harrison and Mark O. Riedl\\
%\affaddr{?}\\
%\affaddr{?}\\
%\affaddr{all over the fucking place}\\
%\email{many}
}

\toappear{}

\maketitle
\begin{abstract}
AI-based games are good.
We made {\sc Contrabot} and {\sc What Did You Do?}.
\end{abstract}

\category{Applied Computing}{Computers in other domains}{Personal computers and PC applications}[Computer games]

\terms{Design}

\keywords{Game AI, design patterns}

%%%%%%%%%%%%%%%%%%%%%%%%%%%%%%%%%%%%%%%%%%%%%%%%%%%%%%%%%%%%%

\section{Introduction}

\noindent AI-based games, woo!

\cite{smith2012:endlessweb}

\bibliographystyle{abbrv}
\bibliography{../latex/lib}

\end{document}